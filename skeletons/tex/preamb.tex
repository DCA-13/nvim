\DeclareMathOperator{\id}{id} % Aplicación identidad
\DeclareMathOperator{\Ima}{Im} % Imagen
\DeclareMathOperator{\Ker}{Ker} % Núcleo
\DeclareMathOperator{\codim}{codim} % Codimensión
\DeclareMathOperator{\Crit}{Crit} % Puntos críticos
\DeclareMathOperator{\Ind}{Ind} % Índice

\let\Oldemptyset\emptyset

\newcommand{\vemptyset}[1]{%
  \scalebox{#1}{\hspace*{-0.25em}\raisebox{-2pt}{
      \begin{tikzpicture}[line cap=round]
        \draw[line width=0.5pt] (0,0) circle [radius=3pt];
        \draw[line width=0.5pt] (-2pt,-5pt) -- (2pt,5pt);
\end{tikzpicture}\hspace*{0.09em}}}}

\DeclareDocumentCommand{\emptyset}{}{{
    \ifcase\mathstyle
      % 0 = \displaystyle % 1 = \crampeddisplaystyle
      \expandafter\vemptyset{1} \or \expandafter\vemptyset{1} \or
      % 2 = \textstyle % 3 = \crampedtextstyle
      \expandafter\vemptyset{1} \or \expandafter\vemptyset{1} \or
      % 4 = \scriptstyle % 5 = \crampedscriptstyle
      \expandafter\vemptyset{0.67} \or \expandafter\vemptyset{0.67}
    \else
      % all other styles
      \vemptyset{0.45}
    \fi
}}

\DeclareDocumentCommand{\quot}{m O{0.5} m O{-0.5}}{% \quot{#1}[#2]{#3}[#4] -> #1/#3
    \ifcase\mathstyle
        % 0 = \displaystyle
        \expandafter
  \setbox0=\hbox{\ensuremath{#1}}% Guarda numerador
  \setbox1=\hbox{\ensuremath{\diagup}}% Guarda barra /
  \setbox2=\hbox{\ensuremath{#3}}% Guarda denominador
  \raisebox{#2\ht1}{\usebox0}% Numerador
  \mkern-5mu\rotatebox{-44}{\rule[#4\ht2]{0.4pt}{0.6cm}}% Barra % debería tener -#4\ht2+#2\ht0+\ht0 en lugar de 0.6cm
  \mkern-4mu%
  \raisebox{#4\ht2}{\usebox2}% Denominador
    \or
        % 1 = \crampeddisplaystyle
        \expandafter
  \setbox0=\hbox{\ensuremath{#1}}% Guarda numerador
  \setbox1=\hbox{\ensuremath{\diagup}}% Guarda barra /
  \setbox2=\hbox{\ensuremath{#3}}% Guarda denominador
  \raisebox{#2\ht1}{\usebox0}% Numerador
  \mkern-5mu\rotatebox{-44}{\rule[#4\ht2]{0.4pt}{0.6cm}}% Barra % debería tener -#4\ht2+#2\ht0+\ht0 en lugar de 0.6cm
  \mkern-4mu%
  \raisebox{#4\ht2}{\usebox2}% Denominador
    \else
        % all other styles
        #1/#3
    \fi
}
